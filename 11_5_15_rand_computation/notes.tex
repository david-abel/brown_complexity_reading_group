%==============================================================================
% Notes for CSCI 2951-F (Fall 2015)
%==============================================================================
\newcommand{\myname}{David Abel}

%==============================================================================
% Mini Preamble.
%==============================================================================

\documentclass[11pt]{article} % 10pt article, want AMS fonts.
\makeatletter					% Make '@' accessible.
\pagestyle{myheadings}				% We do our own page headers.
\def\@oddhead{\bf Randomized Computation \hfill \myname} % Here they are.
\skip\footins=4ex				% Space above first footnote.
\hbadness=10000					% No "underfull hbox" messages.
\makeatother					% Make '@' special again.
\input{../p.tex}

%==============================================================================
% Title.
%==============================================================================

\begin{document}
\centerline{\LARGE{\bf Complexity Reading Group 11/5/2015}}
\vspace{2mm}
\centerline{\Large {\bf Brown University}}

% SECTION: BPP
\section{BPP}

\thm{Definition}{1}{For a function $T: \mathbb{N} \mapsto \mathbb{N}$, and a language $L \subseteq \{0,1\}^*$, consider a Probabilistic Turing Machine (PTM), has two functions $\delta_1$ and $\delta_2$ that each specify a set of rules for the TM. At each step, the PTM flips a fair coin, and the result determines if the PTM uses $\delta_1$ or $\delta_2$ for its current rule. We say a PTM $M$ decides $L$ in time $T(n)$ for every $x \in \{0,1\}^*$, $M$ halts in $T(|x|)$, and $\Pr(M(x) = L(x)) \geq \frac{2}{3}$.}

\thm{Definition}{2}{The class \textsc{BPTime(T(n))}, is the class of languages that can be decided in $\mathcal{O}(T(n))$ by a PTM}

\thm{Definition}{3}{The class \textsc{BPP} $= \cup_c \textsc{BPTime}(n^c)$ is the class of languages that can be decided in polynomial time by a PTM}

\thm{Definition}{4}{A Language $L$ is in \textsc{BPP} if $\exists$ a poly time TM $M$ and a poly $p : \mathbb{N} \mapsto \mathbb{N}$ s.t.:
\begin{equation}
\forall_u : u \in \{0,1\}^*,\ \Pr_r \in \{0,1\}^{p(x)} \left[ M(x,r) = L(x) \right] \geq \frac{2}{3}
\end{equation}}

\elem{Note:} $P \subseteq BPP$, since we could set $\delta_1 = \delta_2$.

\midline

\section{Examples} 

\elem{Example:} Finding the median in a list, i.e. compete the $\frac{n}{2}$-th smallest element in a list.

There is a nice randomized algorithm that can compute the median in $\mathcal{O}(n)$ time.

More general solution: Finding the $k$-th smallest number. 

\elem{Algorithm:} $FindKthSmallest(L,k)$
\begin{itemize}
\item Pick a random index, $i \in [1:len(L)]$
\item Go through the list and put all $a_j \leq a_i$ into $T$.
\item if $|T| \geq k$, then we know the number we're looking for is in $T$. So, return $FindKthSmallest(T, |T| - k)$, or something. The $k$ passed in here isn't exact but it's something like that..
\item if $|T| < k$, then we know it's not in $T$, so it must be in the remainder of $L$ (that we didn't put in $T$). So, return $FindKthSmallest(L \ T, k - |T|)$
\end{itemize}

\midline

\elem{Polynomial Identity Testing:} Suppose we have two polynomials: $P, Q : (x_1, \ldots, x_n) \mapsto\ F$, for some ring $F$. The question is, are $P$ and $Q$ equivalent polynomials?

\elem{Zero Testing:} For a polynomial $P$, is $P = 0$? These two problems are actually identical, because we can ask $P - Q = 0$, and ask $P = Q$, where $Q = 0$.

\thm{Lemma}{1}{Let $p(x_1, \ldots, x_n)$ be non-zero polynomial of total degree at most $d$. Let $S$ be a set of integers with at least $d+1$ elements. If $a_1, \ldots a_n$ are randomly chosen from $S$, then the probability that, if you evaluate $p$ on these integers, then:
\begin{equation}
\Pr(a_1, \ldots, a_n \neq 0) \geq 1 - \frac{d}{|S|}
\end{equation}}

% SECTION: The Class RP
\section{The Class $RP$}


\thm{Definition}{4}{$\textsc{Rtime}(T(n))$ contains any language $L$ for which there is a PTM $M$ that runs in time $T(n)$ such that:
\begin{align}
x \in L \rightarrow \Pr(M(x) = 1) \geq \frac{2}{3} \\
x \not \in L \rightarrow \Pr(M(x) = 1) = 1
\end{align}}



\thm{Definition}{5}{The class $RP$ is the set of all languages that can be decided in polynomial time by an $RTM$:
\begin{equation}
RP = \cup_c \textsc{Rtime}(n^c)
\end{equation}}

\thm{Definition}{6}{For a PTM $M$ and input $X$, define a random variable $T_{M,x}$ as the running time of $M$ over $x$. $\Pr(T_{M,x} = T) = p$ over random choices of $M$ over $x$, it will halt $T$ steps. We say $M$ has expected running time $T(n)$ if $\mathbb{E}\left[T_{M,x}\right] \leq T(|x|)$. Then $M$ has {\it zero sided error}}

\vspace{5mm}

\thm{Definition}{7}{\textsc{ZTime}$(T(n))$ is the class of languages that can be decided with {\it zero sided error}}

\thm{Definition}{8}{$ZPP$ is the class:
\begin{equation}
ZPP = \cup_c \textsc{ZTime}(n^c)
\end{equation}}

\midline


\thm{Lemma}{2}{$ZPP = co-RP \cap RP$}

\elem{Proof:} 

\elem{(a)} WTS: $L \in ZPP$ iff there exists a poly time PTM $M$ with outputs, in $\{0,1?\}$ such that: $\forall_x : x \in \{0,1\}^*$, with probability $1$, $M(x) \in \{ L(x), ? \}$, and $\Pr(M(x) = ?) \leq \frac{1}{2}$.





\elem{(b)} WTS: $ZPP = co-RP \cap RP$











\end{document}