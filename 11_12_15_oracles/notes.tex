%==============================================================================
% Notes for CSCI 2951-F (Fall 2015)
%==============================================================================
\newcommand{\myname}{David Abel}

%==============================================================================
% Mini Preamble.
%==============================================================================

\documentclass[11pt]{article} % 10pt article, want AMS fonts.
\makeatletter					% Make '@' accessible.
\pagestyle{myheadings}				% We do our own page headers.
\def\@oddhead{\bf Oracles \hfill \myname} % Here they are.
\skip\footins=4ex				% Space above first footnote.
\hbadness=10000					% No "underfull hbox" messages.
\makeatother					% Make '@' special again.
\input{../p.tex}

%==============================================================================
% Title.
%==============================================================================

\begin{document}
\centerline{\LARGE{\bf Complexity Reading Group 11/12/2015}}
\vspace{2mm}
\centerline{\Large {\bf Brown University}}

% SECTION: ORACLES
\section{Oracles}

\thm{Definition}{1}{An {\it Oracle Turing Machine} $M$ has:
\begin{itemize}
\item Read/Write Tape (Oracle Tape)
\item Special states ($q_{query}$, $q_{yes}$, $q_{no}$)
\item Need to specify the Oracle itself, $O$, that is, which language $O$ decides.
\end{itemize}}

Execution of $M^O$:
\begin{enumerate}
\item Move to state query
\item If $q \in O$, then next state is $q_{yes}$.
\item If $q \not \in O$, then next state is $q_{no}$.
\end{enumerate}

$M^O$ is a complete machine, meaning it specifies a full computation.

\thm{Definition}{2}{An {\it Oracle Complexity Class} $C$:
\begin{equation}
C = \{ L(M) \mid M \text{ is a $TM$ satisfying } P(M)\}
\end{equation}

Where $P(\cdot)$ is an arbitrary complexity class.

\begin{equation}
C^O = \{ L(M^O) \mid M^O \text{ is an Oracle $TM$ satisfying } P(M^O)\}
\end{equation}
}

If $C$ and $D$ are complexity classes, then:
\begin{equation}
C^D = \cup_{o \in D} C^O
\end{equation}


% Subsection: Examples
\subsection{Examples}

Recall the polynomial hierarchy:
\begin{align*}
\Sigma_{i+1}^P &= NP^{QSAT_i} = NP^{\Sigma_i^P} = NP^{\Pi_i^P} \\
\Pi_{i+1}^P &= (coNP)^{QSAT_i} = (coNP)^{\Sigma_i^P}
\end{align*}


% Section: Relativization
\section{Relativization}

\thm{Definition}{3}{An inclusion $C \subseteq D$ is said to relativize if:
\begin{equation}
\forall_O : C^O \subseteq D^O
\end{equation}}

The notion is also (informally) extended to proofs of inclusions.


% Subsection: Exercises
\subsection{Some properties}

\begin{enumerate}
\item True or False: $C = D \rightarrow \forall_A C^A = D^A$. [{\bf False}]
\item True or False: $C \neq D \rightarrow \forall_A C^A \neq D^A$. [{\bf False}]
\item Some properties:
\begin{itemize}
\item $C \subseteq C^O$
\item $P^O \subset NP^O$
\end{itemize}
\item $D = PH$, then $P^D = NP^D$
\item Choose a random oracle $O$ s.t. $P(x \in O) = \frac{1}{2}$, what is the probability that $P^O \subset Recursive$? Since there are uncountably many undecidable languages and only countably many decidable languages, with probability 1 we get an oracle that decides an undecidable language, so with probability $O$ $P^O \subseteq Recursive$.
\end{enumerate}



% SECTION: Relativization Barrier
\section{Relativization Barrier}

\thm{Theorem}{1  (Baker Gill Solvoy)}{There are oracles $A, B$, s.t. $P^A = NP^A$, but $P^B \neq NP^B$}

\elem{Proof:} 

% Part A
\encircle{Part A:} $A = EC : \{(M, x, 1^n) \mid M(x) = 1 \text{ in $2^n$ steps}\}$ (note: $n$ is an arbitrary parameter).

Therefore $EXP \subseteq P^A$, since you can fix $n$ to be whatever. So $NP^A \subseteq EXP$.

Therefore $P^A = NP^A$. \qed

% Part B
\encircle{Part B:} Given an artbitrary $B$, we know for $U_B = \{1^n \mid \exists x \in B : |X| = n\}$, $U_B \in NP^B$

Now we want $B$ s.t. $U_B \not \in P^B$.

\underline{Diagonlization Argument}

$\{M_i\}: OTM$.


Sort of an inductive gig on $i$. At the `previous' stage we have determined a finite set of strings $\{x\}$ of whether $x \in B$ or not.

For step $i$: run $M_i$ with input $1^{n_i}$, where $n_i > |x| \forall_x : $ s.t. $x$ is determined, for $\frac{2^n}{10}$ steps.

Query: ($q$)
\begin{itemize}
\item If $q$ is determined before, answer consistently.
\item If $q$ is new, answer false.
\end{itemize}

After this execution, we get some result:
\begin{itemize}
\item If $M_i(1^{n_i}) = 1$, then $x \not \in B, \forall_x : |x| = n_i$
\item If $M_i(1^{n_i}) = 0$, then choose some unqueried $x : |x| = n$ and make $x \in B$.
\end{itemize}

$\therefore$ We know that $U_B \not \in DTIME^B(\frac{2^n}{10})$

$\therefore$ $U_B \not \in P^B$ \qed

% SECTION: Other Results
\section{Other Results}

\underline{Proposition:} Draw a random oracle $O$ s.t. for all $n$, $P( \text{ no } |x| = n \in O) = 1/2$, $P(a \text{ uniformly random } x \in O) = 1/2$.

With probability 1 $P^O \neq NP^O$.


\thm{Theorem}{2 (Bennett and Gill)}{Draw a random oracle $O$ s.t. $Pr(x \in O) = \frac{1}{2}$. With probability $1$, $P^O \neq NP^O$.}


\thm{(False) Hypothesis}{1 (Random Oracle Hypothesis)}{Let $S$ be a complexity theoretic statement. Then $S$ is true iff, with probability $1$, $S^A$ is also true, where $A$ is draw as above ($Pr(x \in O) = \frac{1}{2}$).}

Note: The above was disproven!

\thm{Theorem}{3 (Chang et. al. 1994)}{With probability $1$, $IP^A \neq PSPACE^A$}


\end{document}