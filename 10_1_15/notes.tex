%==============================================================================
% Notes for CSCI 2951-F (Fall 2015)
%==============================================================================
\newcommand{\thishw}{\bf Complexity Reading Group 10/1/2015}
\newcommand{\myname}{David Abel}

%==============================================================================
% Mini Preamble.
%==============================================================================

\documentclass[11pt]{article} % 10pt article, want AMS fonts.
\makeatletter					% Make '@' accessible.
\pagestyle{myheadings}				% We do our own page headers.
\def\@oddhead{\bf Circuits \hfill \myname} % Here they are.
\skip\footins=4ex				% Space above first footnote.
\hbadness=10000					% No "underfull hbox" messages.
\makeatother					% Make '@' special again.
\input{p.tex}


%==============================================================================
% Title.
%==============================================================================

\begin{document}
\centerline{\LARGE\thishw}

% -- SECTION: Circuit --
\section{Circuit}

\begin{itemize}
\item $C_n$ n-input circuit
\item DAG n sources and 1 sink
\item All non sources labelled with $\wedge, \vee \neg$
\item $|C|$ = number of vertices
%\item Assuming $f$ input bits, there will be $f-1$ gates.
\end{itemize}

% -- SECTION: Circuit Families --
\section{Circuit Families}

Different circuit depending on what $n$ is in. Specifcally, have a function $T(n)$ that determines the size of the circuit. $\forall_n : |C_n| \leq T(n)$.

\thm{Definition}{1}{A language $\mathcal{L}$ is in \textsc{Size}$(T(n))$ if $\exists : T(n)$-sized circuit family s.t. $\forall_x \in \{0,1\}^n x \in \mathcal{L} \equiv C_n (x) = 1$.}

\thm{Definition}{2}{$P_{\text{poly}}$ is the class of languages decidable by polynomially sized circuit families:
\begin{equation*}
P_{\text{poly}} = \cup_c \textsc{Size}(n^c)
\end{equation*}}


\elem{Claim:} $p \subseteq P_{\text{poly}}: \mathcal{L} \in P \rightarrow P_{\text{poly}}$, where $\mathcal{L}$ is a language, $P$ is the complexity class.

wts: $\forall : T(n)$-time Turing Machines $M$< $\exists : \mathcal{O}(T(n))$ sized circuit fam:
\begin{equation*}
\{C_n\}_{m \in \mathbb{N}} : C_n(x) = M(x), \forall_x : x \in \{0,1\}^n
\end{equation*}

\thm{Definition}{3}{Oblivious Turing Machine is a machine where head movements depend on $|x|$ but not on contents of $x$.}

\subsection{Proof about circuit family relation to TM}

Intuition: the class of languages decidable by polynomial sized circuit families is a superset of $P$ (polynomial time TMs).

\thm{Lemma}{1}{Given a TM $M$ that decides $L$ in $t(n)$ time $\exists$ oblivious TM $M'$ that decides $L$ in $\mathcal{O}(t(n)^2)$ time, (and uses 2 tapes).}


\elem{Proof:} For any input $x \in \{0,1\}^n$, define transcript of $M$'s execution to be: $z_1, \ldots, z_{T(n)}$, where $z_i$ denotes what's happening at step i:
\begin{itemize}
\item input read by each head
\item current state of the TM (constant number of states)
\end{itemize}

Note: $z_i$ depends on $z_{i-1}, z_{i_1}, z_{i_2}$, where $z_{i_n}$ is the last time step where head $h$ was at the same position it's at in step $i$.

$\therefore \exists : $ a constant sized circuit $C_i$ representing $z_i$'s dependance on $z_{i-1}, z_{i_1}, z_{i_2}$.

Now, chain together the $C_i$'s for $i=1,\ldots,T(n)$.

This creates a circuit of size $\mathcal{O}(T(n))$.

Add a constant number of additional gates to determine if we're in an accept state. \qed

% -- SECTION: Uniform vs. Non-Uniform Circuits --
\section{Uniform vs. Non-Uniform Circuits}

$\textsc{Halt} = \{1^n \mid n$ encodes TM, input pairs $\langle M, x  \rangle : M$ halts on x$ \}$

Non-uniform circuit families contain the language $\textsc{Halt}$.

\thm{Definition}{4}{A uniform circuit is one where the circuits can be constructed by a poly-time TM. This defines the class $P$-uniform.}

\elem{Some other results:}
\begin{enumerate}
\item $L$ can be decided by a $P$-uniform circuit family $\equiv L \subseteq P$
\item $P_{\text{poly}}$ = $P + $ ``advice"
\end{enumerate}


\thm{Definition}{5}{Class of language decidable by $T(n)$ TM's equiv with $a(n)$ advice \textsc{Dtime}$(T(n))/a(n)$}


\elem{Claim:} $P_{\text{poly}} = \cup_{c,d} \frac{\textsc{Dtime}(n^c)}{n^d}$

\elem{Proof:}

First direction ($\rightarrow$): $L \in P_{\text{poly}} : $ let $\alpha_n$ be description of $C_n$

Second direction ($\leftarrow$): $\exists : M (x, \alpha_n)$, hard code $\alpha_n$, use same transformation as $P \subseteq P_{\text{poly}}$ proof.

% Subsection: Karp, lipton theorem.
\subsection{Karp, Lipton Theorem '80}

\thm{Theorem}{2}{$NP \subseteq P_{\text{poly}} \rightarrow PH = \Sigma_2^P$}


% -- SECTION: Polynomial Hierarchy --
\section{Polynomial Hierarchy}

\thm{Definition}{6}{For $i \geq 1$, $L \in \Sigma_i^P$ if $\exists$ poly-time TM, $M$, and a polynomial $q$, s.t.:
\begin{equation*}
x \in L \equiv \exists_{u_1} : u_1 \in \{0,1\}^{q(|x|)} \forall_{u_2} : u_2 \in \{0,1\}^{q(|x|)} \ldots Q_i u_i \in \{0,1\}^{q(|x|)}
\end{equation*}

Where $Q_i = \begin{cases}
\forall& i= 0 \mod 2 \\
\exists& i=1 \mod 2
\end{cases}$

\begin{align*}
\text{Polynomial Hierarchy} (PH) &= \cup_i \Sigma_i^P \\
\Pi_i^P &= co\Sigma_i^P = \{\bar{L} : L \in \Sigma_i^P\}
\end{align*}
\begin{align*}
coNP &= \Pi_i^P \\
NP &= \Sigma_i^P
\end{align*}

So:
\begin{equation*}
\Sigma_i^P \subseteq \Pi_{i+1}^P \subseteq \Sigma_{i+2}^P \rightarrow PH = \cup_i \Pi_i^P
\end{equation*}
}

% -- SECTION: Summary --
\section{In Summary:}

Circuits: Covered 6.1, 6.2, 6.3 of the book.

Polynomial Hierarchy: 5.2.

For next time:
More polynomial hierarchy, theorem 5.4.
Sasha: office 421

\end{document}